\documentclass{beamer}
\usepackage[utf8]{inputenc}
\usepackage{listings}
\usepackage{xpatch}
\usepackage{hyperref}
\usepackage{xcolor}
\usepackage{listings}
\usepackage{realboxes}
\usetheme{AnnArbor}
\usecolortheme{beaver}
\setbeamercolor{titlelike}{parent=structure,bg=white}
\setbeamercolor{itemize item}{fg=black}
\setbeamercolor{itemize subitem}{fg=black}
\setbeamercolor{itemize subsubitem}{fg=black}

\setbeamertemplate{itemize item}[square]
\setbeamertemplate{itemize subitem}[circle]
\setbeamertemplate{itemize subsubitem}[triangle]

\setbeamercolor{block title}{bg=black!5,fg=black}

\definecolor{mygray}{rgb}{0.8,0.8,0.8}
\lstset{
  basicstyle=\ttfamily,
  backgroundcolor=\color{mygray},
}

\makeatletter
\xpretocmd\lstinline{\Colorbox{mygray}\bgroup\appto\lst@DeInit{\egroup}}{}{}
\makeatother

\setbeamertemplate{footline}
{
  \leavevmode%
  \hbox{%
  \begin{beamercolorbox}[wd=.4\paperwidth,ht=2.25ex,dp=1ex,center]{author in head/foot}%
    \usebeamerfont{author in head/foot}Sistemas operativos
  \end{beamercolorbox}%
  \begin{beamercolorbox}[wd=.6\paperwidth,ht=2.25ex,dp=1ex,center]{title in head/foot}%
    \insertframenumber{} / \inserttotalframenumber\hspace*{1ex}
  \end{beamercolorbox}}%
  \vskip0pt%
}

\title{GDB guide}
\author{jorge.gonzalez, martin.carrasco}
\date{\today}

\begin{document}

\maketitle

\begin{frame}{What is GDB?}
    \begin{itemize}
        \item<1-> It's the GNU Project Debugger
        \item<2-> Allows you to analyze a running program
        \item<3-> Print variable names, change variable values, stop at certain conditions, etc.
    \end{itemize}
\end{frame}

\begin{frame}{How to use GDB}
	\begin{itemize}
    	\item<1-> Start GDB session for a program: \lstinline{gdb <program>}
    	\item<2-> Start program execution \lstinline{run}
    	\item<3-> Setup  a breakpoint where execution will stop \lstinline{break <file>:<line>} \lstinline{disable <num>} \lstinline{info <breakpoint>}
    	\item<4-> Execute current line and break at next \lstinline{next}
    	\item<5-> Run a trace of the calls that led to current point \lstinline{backtrace}
   		\item<6-> Print a variable content \lstinline{print <var>} \lstinline{set <expression>}
    \end{itemize}
\end{frame}

\begin{frame}{Configurations for GDB}
	\begin{itemize}
        \item Install GDB
    	\item Define GDBMacros variable in the local enviroment
        \item Run docker container (if you are using one ) with \lstinline{docker run --cap-add=SYS_PTRACE --security-opt} \par \lstinline{seccomp=unconfined}
	\end{itemize}
\end{frame}

\begin{frame}{GDB in PintOS}
	\begin{itemize}
		\item<1-> \textbf{Shell1}: Start PintOS \lstinline{pintos --gdb -- run mytest}
		\item<2-> \textbf{Shell2}: Load binary \lstinline{pintos-gdb kernel.o}
        \item<3-> \textbf{Shell2}: Attach to proccess \lstinline{target remote localhost:1234}
        \item<4-> Start debbuging by pressing \lstinline{c}
	\end{itemize}
\end{frame}

\begin{frame}{Configuration for GDB: Graphical}
    \begin{figure}
        \centering
        \includegraphics[scale=0.2]{attach.png}
    \end{figure}
\end{frame}


\begin{frame}{Usefull PintOS-GDB macros}
	\begin{itemize}
    	\item \textbf{btthread thread}: Run a backtrace on a specific thread
    	\item \textbf{dumplist list-name list-type element}: Print the element of a given list
    	\item \textbf{hook-stop}: This is called everytime a pagefault occurs
    	\item \textbf{btthreadlist list-name element}: Shows the backtrace of all the threads contained in the list
    	\item \textbf{btpagefault}:Backtrace of a thread after a page fault.
	\end{itemize}
\end{frame}

\begin{frame}{Usage example: Alarm Clock}
    Live example
\end{frame}

\begin{frame}{Bibliography}
    \href{https://beej.us/guide/bggdb/}
    \href{https://web.stanford.edu/class/cs140/projects/pintos/pintos.pdf}
\end{frame}

\end{document}
